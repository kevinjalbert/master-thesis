\chapter{Introduction}
\label{chap:introduction}
Mutation testing has traditionally been used as a coverage technique to evaluate the effectiveness of test suites and provide confidence in the testing process~\cite{DLS78, JH10}. For over 30 years, mutation testing has been applied to software written in programming languages including C~\cite{DM96, JH08}, Fortran~\cite{KO91} and Java~\cite{MKO02, BCD06}. Furthermore, mutation testing has also been applied to non-programming artifacts such as formal specification languages~\cite{ABM98} and spreadsheets~\cite{AE09}.

Mutation testing uses a set of \emph{mutation operators} to generate faulty versions of a program called \emph{mutants}. Mutation operators are created based on an existing fault taxonomy and each operator usually corresponds to a specific type of fault. A test suite is evaluated against a set of mutants to determine the \emph{mutation score}. The mutation score is defined as the percentage of non-equivalent mutants that are detected (\emph{killed}) by a test suite. The better a test suite, the more mutants will be killed and thus the higher the mutation score.

A major drawback of mutation testing is that even a small program may yield hundreds or thousands of mutants -- potentially making the process cost prohibitive in comparison to other coverage metrics. Three approaches have been proposed to improve mutation testing performance and scalability~\cite{OU00}:

\begin{enumerate}
  \item \textbf{``Do fewer'' approach:} this category of optimizations aim to decrease the computational cost of mutation testing by reducing the number of mutants that a test suite is evaluated against. The most popular example from this category is selective mutation -- the use of a subset of mutation operators that have been empirically shown to be as effective as using an entire set of operators~\cite{OLR+96}.

  \item \textbf{``Do smarter'' approach:} this category of optimizations aim to decrease the cost of mutation testing by improving the actual mutation testing technique. For example, weak mutation \emph{``...is an approximation technique that compares the internal states of mutant and original program immediately after execution of the mutated portion of the code (instead of comparing the program output)''}~\cite{OU00}.

  \item \textbf{``Do faster'' approach:} this category of optimizations aim to reduce the cost of mutation testing by focusing on performance. For example, one ``do faster'' approach improves compilation time using schema-based mutation -- \emph{``...encodes all mutations into one source level program...''}~\cite{OU00}.
\end{enumerate}

As an alternative to the above approaches, we propose a ``do fewer and smarter'' technique for mutation testing at the unit level.  When mutation testing is used for the creation or improvement of a test suite,  the test suite will often have to be applied to the mutants in an iterative fashion as tests are added, removed and modified. Furthermore, the effects on the mutation score after each iteration have to be observed. We propose to replace at least some of the mutation testing of intermediate tests with mutation score prediction and thus decrease the number of mutants that have to be evaluated using a test suite. Our proposed approach uses machine learning to predict the mutation score based on a combination of source code and test suite metrics of the code unit under test.

Next, in Chapter~\ref{chap:background} we describe the machine learning technique, the mutation testing tool and the metrics gathering tools used in our research. In Chapter~\ref{chap:approach} we describe our overall approach to mutant score prediction. In Chapter~\ref{chap:experiment} we apply our approach to an open source project and discuss the effectiveness of our prediction based on this preliminary evaluation. Finally, in Chapter~\ref{chap:related_work}~\&~\ref{chap:conclusion} we discuss related work, conclusions, and future work.
