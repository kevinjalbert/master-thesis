% !TEX root = thesis.tex
\addcontentsline{toc}{chapter}{Abstract}
\chapter*{Abstract}
Mutation testing has traditionally been used to evaluate the effectiveness of test suites and provide confidence in the testing process. Mutation testing involves the creation of many versions of a program each with a single syntactic fault. A test suite is evaluated against these program versions (mutants) in order to determine the percentage of mutants a test suite is able to identify (mutation score). A major drawback of mutation testing is that even a small program may yield thousands of mutants and can potentially make the process cost prohibitive. To improve the performance and reduce the cost of mutation testing, we propose a machine learning approach to predict mutation score based on a combination of source code and test suite metrics. We conduct a number of experiments using 8 open source projects to evaluate the effectiveness of our approach. We were able to achieve an average method-level prediction accuracy of 49.7920\% using our 8 experimental subject. The pair of configuration parameters were also determined through experimentation to achieve the reported accuracy without per-project tuning. Finally we determined that about \sfrac{1}{3} of the mutation testing data is required for training to achieve the reported level of accuracy.