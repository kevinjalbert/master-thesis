% !TEX root = presentation.tex
\section{Introduction}
\subsection{Motivation}
\frame{\frametitle{Motivation}
  \begin{itemize}
    \item \alert{Software testing and verification} is costly (approximate cost for United States in 2002 was \alert{\$59.5 billion})~\footcite{RTI02}.
    \item \alert{Core} software testing \alert{artifacts}:
    \begin{itemize}
      \item Source code.
      \item Test suite.
    \end{itemize}
  \end{itemize}
}

\subsubsection{Test Suite Effectiveness}
\frame{\frametitle{Test Suite Effectiveness}
  \begin{itemize}
    \item An effective test suite is \textit{``\ldots one that is capable of \alert{detecting all real bugs}''}~\footcite{Wey93}.
    \item Can measure \alert{code coverage} (e.g., branch, statement, path) being exercised by a test suite~\footcite{ZHM97}
    \item A more \alert{effective} determination of test suite effectiveness is \alert{mutation testing}.
  \end{itemize}
}

\subsubsection{Mutation Testing}
\frame{\frametitle{Mutation Testing}
  \begin{itemize}
    \item \alert{Mutation operators}:
    \begin{itemize}
      \item Generate faulty versions (i.e., \alert{mutants}) of a software system.
      \item Based on existing \alert{fault taxonomy}.
    \end{itemize}
    \item Mutants are \alert{tested} against a test suite.
    \item Percentage of detected mutants (i.e., \alert{killed}) represents the \alert{mutation score}.
  \end{itemize}
}

\subsection{Problem}
\frame{\frametitle{Problem}
  \begin{itemize}
    \item \alert{Hundreds or thousands} of mutants can be generated for even \alert{small} software systems.
    \item \alert{Slow industry adoption} of mutation testing due to \alert{performance/scalability} in evaluating mutants~\footcite{OU01}.
    \item Mutation testing would have to be applied \alert{frequently} during development
  \end{itemize}
}

\subsection{Thesis Statement}
\frame{\frametitle{Thesis Statement}
  \begin{large}
    \textit{``The use of source code and test suite \alert{metrics} in combination with \alert{machine learning} techniques can accurately predict \alert{mutation scores}. Furthermore, the predictions can be used to \alert{reduce} the performance \alert{cost} of mutation testing when used to iteratively develop test suites.''}
  \end{large}
  \vspace{5mm}
  \hrule
  \vspace{5mm}
  \begin{itemize}
    \item A \alert{``do fewer and smarter''} technique for mutation testing.
    \begin{itemize}
      \item Identify source code units that have \alert{low/high coverage}.
      \item Ability to \alert{prioritize} mutation testing for specific mutants.
    \end{itemize}
  \end{itemize}
}
