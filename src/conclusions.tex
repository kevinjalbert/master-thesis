% !TEX root = thesis.tex
\chapter{Summary and Conclusions}
\label{chap:conclusions}
We present a complete summary of the thesis in Section~\ref{sec:conclusions_summary}. We indicate the contributions this thesis has accomplished in Section~\ref{sec:conclusions_contributions}. In Section~\ref{sec:conclusions_limitations} we raise a number of concerns and limitations as well as how we mitigated these where possible. We outline alternative steps in our approach and experiments as well as additional future work in Section~\ref{sec:conclusions_future_work}. Finally we conclude this thesis in Section~\ref{sec:conclusions_conclusions}.


\section{Summary}
\label{sec:conclusions_summary}
Mutation testing is a resource intensive process as there is potentially thousands of mutants (i.e., versions of \gls{sut} with a single syntactic change that might induce a fault) that can be produced for each \gls{sut}. Mutation testing generates a set of mutants from the source code of the \gls{sut} and than evaluates these using the provided test suite. A mutation score is calculated as a result of mutation testing, which in indicates how effective the test suite is at finding faults (i.e., test suite effectiveness).

Oppose to an improving mutation testing performance by adjusting the mutation testing process (i.e., better mutation representation/generation/evaluation) we decided to apply machine learning to make predictions for the mutation score of source code units. We consider the source code and test suite as strong sources of features/attributes for our predictions as they are directly involved in the mutation testing process. As described in Chapter~\ref{chap:approach} we use a \gls{svm} to make predictions based on the features of class- and method-level source code units. We identify 4 initial sets of metrics (i.e., feature sets) from the \gls{sut}: source code, coverage, accumulated source code and accumulated test suite.

We outline our 8 experimental subjects that contained about 1689 classes, 113686 method and 10233 test cases in Section~\ref{sec:experiment_subjects}, in which the total time required for mutation tested was about 64.5 hours. Our approach was able to collect 864 class-level and 5510 method-level source code units that were covered by mutation testing and fit our criteria for training and prediction. We observed the mutation score distribution of all covered source code units in Section~\ref{subsec:experiment_mutation_score_distribution} and determined a three category approach to abstract the real-values of mutation scores.

In Section~\ref{subsec:experiment_cross_validation} we used our selected categories we performed cross-validation utilizing all available data with undersampling over the different feature sets. Our results showed that the combination of the feature sets provides a higher cross-validation accuracy. We then observed the prediction accuracy on unknown data in Section~\ref{subsec:experiment_prediction} using the individual subjects and sets of all but a single subject. Our results here showed that class-level mutation score prediction was either harder to predict or missing additional features. Method-level predictions were better than class-level predictions, furthermore training and predicting contained to an individual subject yielded higher accuracy than that of the all but a single subject. Due to limited data the individual prediction accuracy had large variations in prediction accuracy. We identified drawbacks to our prediction experiment in Section~\ref{subsec:experiment_optimization_generalization} that showed using cross-validation accuracy for \gls{svm} parameter selection can be ineffective. We presented several alternative and more effective performance measures for machine learning techniques, namely \emph{F-score}. Using F-score we conduced our own grid search for a single set of \gls{svm} parameters that maximized F-score across all predictions. We identified generalizable \gls{svm} parameters for class- and method-level predictions, and redid our prediction accuracy experiment using the new parameters. We were able to increase the average prediction accuracy by 4.8147 and 3.7847 for class- and method-level source code units respectively. We looked at the implications of data availability and potential usage for iterative development in Section~\ref{subsec:experiment_data}. By limiting the amount of data used for training we were able to find that roughly \sfrac{1}{3} of the data is needed for training to maximize prediction accuracy. 


\section{Contributions}
\label{sec:conclusions_contributions}
Throughout the experiment chapter (see Chapter~\ref{chap:experiment} we noted the following in terms of contributes to the domain of mutation score prediction:

\begin{itemize}
	\item Identified several features that can be used to prediction mutation scores of source code units
	\item Presented the mutation score distribution of covered mutants over 8 open-source systems
	\item Observed the effect on cross-validation accuracy with respect to the feature sets
	\item Observed the effect of predicting on unknown source code units (i.e., not used for training) both within an individual project and across projects
	\item Found a set of \gls{svm} parameters for class- and method-level predictions that maximize generalizability in the observed open-source systems.
	\item Understood the impact on prediction accuracy based on the amount of data from an individual project is used for training.
\end{itemize}

Specifically we showed that using all the available features we were able to achieve an average prediction accuracy of 37.8498\% and 49.7920\%, for class- and method-level source code units respectively. Method-level predictions have a much higher accuracy than class-level predictions, which is all right as method-level source code units are finer grain in terms of scope. We achieved higher than random prediction accuracies using generalized \gls{svm} parameters, which removes the need of finding suitable parameters for new data. Our approach is novel in that we consider both source code and test suite metrics as factors in making the predictions in software testing. We also performed feature selection on the used features in Appendix~\ref{app:feature_selection}, which showed that it is possible to reduce feature set while retaining prediction accuracy.


\section{Limitations \& Threats to Validity}
\label{sec:conclusions_limitations}
As with any study/experiment there will be limitations and threats to validity. In Section~\ref{subset:conclusions_limitations_approach} we discuss limitations with our approach. We consider the four categories for threats to validity with respect to experimentation in Section~\ref{subsec:conclusions_conclusion_validity}--~\ref{subsec:conclusions_external_validity}~\cite{WRH+00,WKP10}.


\subsection{Limitations of Approach}
\label{subset:conclusions_limitations_approach}
Our approach for predicting mutation scores based on source code and test suite metrics utilizes a number of tools. Namely we have: \emph{Javalanche} to collect mutation scores, \emph{Eclipse Metrics Plugin} to collect source code and test suite metrics, \emph{EMMA} to collect additional test suite coverage metrics, and \emph{LIBSVM} to perform the training and prediction of the source code units. We selected tools based on the metrics they could provide as well as the the output format, there might be other candidates tool that could have performed better. In particular the mutation testing tool we selected is not the newest, and omits a whole class of mutations (i.e., class-level object-oriented mutants), which could be misrepresenting the mutation scores. In our approach we had removed any source code unit that was abstract/overloaded/anonymous due to their complexity, which reduced our usable data and could have misrepresented the actual systems. The tools used to collect the features of the source code units might not be comprehensive in terms of features that describe the source code units. 


\subsection{Conclusion Validity}
\label{subsec:conclusions_conclusion_validity}
Threats to conclusion validity involve issues with the process and statistical means to draw any conclusions regarding experiments. We utilized various summary statistical measures to determine the conclusions of our results. In particular we used the following statistical measures: mean, standard deviation, quartiles and frequencies to understand our experiments with respect to their results. Furthermore with our results we conducted a minimum of 10 executions per experiment to mitigate the randomness of our results. With respect to drawing conclusions were comparing the average accuracy to what a random prediction would achieve. Thus by comparing the mean accuracies we were able to compare our approach to random. In retrospect we should have performed more executions per experiment to further reduce the noise. Furthermore we could have performed a statistical test to understand the statistical significants of our comparison.


\subsection{Internal Validity}
\label{subsec:conclusions_internal_validity}
Internal threats to validity are concerned with factors that could influence the independent variable in our experiments. Our independent variables are the features themselves from the 8 open-source systems that we selected. Obviously there could be issues that can arise based on the measures that our tools returned for each system, though these tools are well established and provide simplistic measures (i.e., issues are unlikely to arise due to \emph{incorrect} results). With respect to the mutants themselves that are generated by the \emph{Javalanche} mutation testing tool, the version used was experimental and could be more susceptible to \emph{incorrect} results. Furthermore \emph{Javalanche} uses a subset of mutants (i.e., mainly method-level mutants), which could have a major impact on the class-level source code units. With respect to \emph{true} internal validity the independent variables are not influencing each other in ways that we were not aware of that could be detrimental to our experiment. 


\subsection{Construct Validity}
\label{subsec:conclusions_construct_validity}
Whether the independent and dependant variables we are using actually align with the problem we are experimenting with is an issue with construct validity. In our experiment we are using a set of features extracted from open-source software systems (i.e., the independent variables) to determine the accuracy of predicting mutation score (i.e, the dependant variable). Machine learning performance measures (i.e., accuracy, F-score, etc...) are valid dependant variables as they measure the effectiveness of the classification technique. The independent variables for machine learning are harder to determine by nature, there is often no clear set of features for making predictions. For our experiment we observed the two main components involved in mutation testing, and these are the source code and test suite. These two components can be represented in quantifiable metrics (i.e., source code and test suite metrics) which are commonly known and used in Software Engineering research. 


\subsection{External Validity}
\label{subsec:conclusions_external_validity}
With experiments one of the major concerns involve the ability for the results to generalize outside of the study, that is external validity. With our experiment we specifically avoided toy-problems and opted to use open-source systems, which are real software systems. These systems are not industrial system nor extremely large-scale (i.e., 100K+ \gls{sloc}), thus we are unsure if the results would generalize to such systems. The open-source systems we chose had some variation in domain (i.e., library, framework, etc...) though our set obviously does not act as a representative of different domains. In addition most of the subjects we used had relatively \emph{good} test suites (i.e., of the covered mutants the mutation scores were above 50\%), we are unsure how our prediction would perform on systems with \emph{poor} test suites. Furthermore we observed only the Java language, whether these results generalize to other languages has not been verified. As stated by Kitchenham and Mendes \textit{``It is invalid to select one or two datasets to `prove' the validity of a new technique because we cannot be sure that, of the many published datasets, those chosen are the only ones that favour the new technique''}~\cite{KM09}. We used only 8 open-source projects as our datasets for our prediction technique, even though this is more then one or two it is still quite limited.  Mutants can be influenced by external factors such test suite size and mutation operators as it was found that class-level mutants are harder to detect than traditional method-level mutants~\cite{NK11}. As we used only traditional mutation operators this could have an impact on the generalizability, same with the varying sizes of the test suites of our subjects.


\section{Future Work}
\label{sec:conclusions_future_work}


\section{Conclusions}
\label{sec:conclusions_conclusions}
Our technique for predicting mutation score using source code and test suite metrics outperforms random with an achieved accuracy of 58.27\% and 54.82\% with the JGAP data, for classes and methods respectively. These results have not been optimized and we believe that with further enhancements and a more tailored feature set we may be able to increase the prediction accuracy.

Despite the promising initial results there is an obvious threat to external validity since we have applied our predictive technique to a single open source project -- JGAP. Additionally, we performed training and prediction from the same project. As stated by Kitchenham and Mendes \textit{``It is invalid to select one or two datasets to `prove' the validity of a new technique because we cannot be sure that, of the many published datasets, those chosen are the only ones that favour the new technique''}~\cite{KM09}. Thus, we plan to evaluate more open source projects using our prediction technique to better assess the prediction accuracy. With more data we plan to investigate whether cross-project models are valid for mutation score prediction. We would also like to consider projects with more varied mutation scores to explore the variation in prediction accuracy between strong and weak test suites. A final area of future work is to expand the set of mutation operators used to include object oriented and concurrency operators.
